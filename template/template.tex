% -------------------------------------------------------------------------------------------------
% Archivo: template.tex
% Descripción: Plantilla básica en LaTeX para trabajos académicos
% Autor: Manuel García Cortés
% -------------------------------------------------------------------------------------------------

\documentclass[a4paper, twoside]{article}
\usepackage{estilo}
\graphicspath{{img/}}


% Configuración del documento
\title{Un título espectacular}
\author{Un autor increíble}
\date{\today}

\institute{Departamento de Informática y Automática, Universidad de Salamanca, España. \\ 
Máster universitario en Sistemas Inteligentes: Asignatura XYZ}

% Cabeceras running
\settitlerunning{Título corto}
\setauthorrunning{Autor corto}

\begin{document}
\maketitle

\begin{abstract}
    Un resumen conciso del trabajo que destaca los puntos clave y objetivos principales del estudio realizado.
    
    \keywords{palabras clave, separadas, por, comas}
\end{abstract}


\newpage

% Este bloque genera el índice de contenidos y de figuras
% ---------------------------------------------------------
\newpage
\tableofcontents
\newpage
\listoffigures
\listoftables
\newpage
% ---------------------------------------------------------

\newpage

% Contenido del documento aquí
\section{Introducción}
Aquí comienza la introducción del trabajo académico. Se presentan los objetivos y el contexto del estudio.

\begin{figure}[htbp]
    \centering
    %\fbox{\includegraphics[width=0.95\linewidth]{}}
    \fbox{}
    \caption{Figura Placeholder}
    \label{fig:ClusterK7_b}
\end{figure}

\begin{table}[htbp]
    \centering
    \caption{Table Placeholder.}
    \smallskip
    \label{tab:knn_metrics}
    \begin{tabular}{|l|c|}
        \hline
        \textbf{Métrica} & \textbf{Valor} \\
        \hline
        Accuracy & 0.7690 \\
        Precision & 0.7690 \\
        Recall & 0.7690 \\
        F1-Score & 0.7690 \\
        Cohen Kappa & 0.7306 \\
        \hline
    \end{tabular}
\end{table}

\newpage

% Bibliografía
\begin{thebibliography}{9}
    \bibitem{ref1} Autor, Título del libro, Editorial, Año. (Siguiendo el formato IEEE)
\end{thebibliography}

\end{document}